\section*{Conclusões}
O ransomware permanece uma ameaça cibernética persistente, dinâmica e de alto impacto, evoluindo 
continuamente em suas táticas, alvos e modelos de negócios. Desde o rudimentar AIDS Trojan até as sofisticadas 
operações de RaaS atuais que empregam dupla e tripla extorsão, os atores de ransomware demonstraram uma capacidade 
notável de adaptação e inovação. A análise histórica revela uma trajetória de crescente complexidade técnica, exploração de 
novas vulnerabilidades e uma profissionalização das atividades criminosas.

As técnicas de ataque continuam a explorar tanto falhas tecnológicas quanto o fator humano, com phishing, exploração de 
vulnerabilidades não corrigidas e comprometimento de credenciais permanecendo vetores de infecção proeminentes. A mudança para 
alvos de alto valor, infraestruturas críticas e a expansão para plataformas não-Windows, como Linux e ESXi, demonstram a ambição 
e o alcance crescentes desses grupos. A exfiltração de dados tornou-se um componente padrão, aumentando a pressão sobre as vítimas 
e complicando as estratégias de recuperação baseadas apenas em backups.

As estratégias de proteção devem, portanto, ser abrangentes e multicamadas, integrando rigorosa higiene cibernética, 
conscientização e treinamento contínuo dos usuários, tecnologias avançadas de detecção e prevenção, e, crucialmente, planos 
robustos de backup e recuperação de desastres. A resiliência dos próprios backups, através de imutabilidade e isolamento, 
emergiu como um diferencial crítico, dado o foco dos atacantes em neutralizar essa linha de defesa. A adoção de frameworks de 
segurança como o NIST CSF 2.0 e o desenvolvimento de Planos de Resposta a Incidentes detalhados e testados são fundamentais para 
uma postura de segurança proativa e adaptativa.

A decisão de pagar o resgate continua sendo um dilema complexo, com autoridades desaconselhando veementemente o pagamento, 
enquanto as vítimas enfrentam pressões operacionais e o risco de vazamento de dados. As estatísticas indicam que o pagamento 
não garante a recuperação e pode perpetuar o ciclo de ataques.

Olhando para o futuro, espera-se que o ransomware continue a evoluir, possivelmente alavancando a 
Inteligência Artificial tanto para aprimorar ataques (e.g., phishing mais convincente, desenvolvimento de malware mais evasivo) 
quanto para fortalecer as defesas. A exploração de vulnerabilidades em um espectro cada vez maior de dispositivos 
conectados (IoT, OT) e a contínua sofisticação das táticas de extorsão exigirão vigilância constante. A colaboração internacional 
entre governos, agências de aplicação da lei e o setor privado, juntamente com a partilha de informações sobre ameaças, será 
indispensável para combater eficazmente esta ameaça global. Em última análise, o investimento contínuo em pesquisa, tecnologia, 
treinamento e, acima de tudo, em uma cultura de resiliência cibernética, é essencial para que as organizações possam enfrentar os 
desafios impostos pelo ransomware.