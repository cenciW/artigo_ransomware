\section*{Conclusões}

Devem basear-se exclusivamente nos resultados do trabalho. Evitar a repetição dos resultados em listagem subsequente, buscando, sim, confrontar o que se obteve com os objetivos inicialmente estabelecidos.

\section*{CONTRIBUIÇÕES DOS AUTORES}
Apresente de forma simplificada as contribuições de cada autor. Esta seção é baseada na Taxonomia CRediT e visa descrever as contribuições dos autores no trabalho. Como sugestão, utilize o Guia para Marcação e Publicação de contribuição de autores: Taxonomia CRediT da Scielo, disponível em: \url{https://wp.scielo.org/wp-content/uploads/credit.pdf}.

Exemplo: M.F.C, E.B.M.S e K.L.G. (podem ser utilizadas as iniciais do nome) contribuíram com a concepção e escopo do estudo. H.F.S e M.F.C procederam com a metodologia e experimentos. M.S.L, K.L.G. e E.B.M.S escreveram o trabalho.

Todos os autores contribuíram com a revisão do trabalho e aprovaram a versão submetida.


\section*{Agradecimentos}

Inserir após as conclusões, de maneira sucinta. Se o projeto for financiado por alguma agência de fomento, citar a fonte.