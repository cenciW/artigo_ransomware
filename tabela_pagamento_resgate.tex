\begin{table}[htbp]
\centering
\small
\caption{Prós e Contras do Pagamento de Resgate}
\label{tab:tabela_pagamento_resgate.tex}
\begin{tabular}{|p{4cm}|p{5cm}|p{5.5cm}|}
\hline
\textbf{Argumentos a Favor do Pagamento (Perspectiva da Vítima)} & \textbf{Argumentos Contra o Pagamento (Perspectiva de Segurança/Legal)} & \textbf{Realidade Estatística/Riscos Associados} \\
\hline
Potencial recuperação rápida de dados e sistemas críticos. & Não há garantia de recebimento da chave de decriptografia ou de sua funcionalidade. & Apenas 54\% das vítimas que pagaram em 2024 recuperaram dados (queda de 73\% em 2 anos). Chaves podem ser defeituosas ou incompletas. \\
\hline
Evitar o vazamento público de dados sensíveis exfiltrados (dupla extorsão). & Os dados já podem ter sido copiados, vendidos ou vazados antes ou mesmo após o pagamento. Pagar não garante o sigilo. & Criminosos podem não cumprir a promessa de não vazar os dados. \\
\hline
Retomar rapidamente as operações comerciais e minimizar perdas financeiras por inatividade. & O pagamento financia atividades criminosas e incentiva futuros ataques contra a própria organização e outras. & O custo total do incidente geralmente excede em muito o valor do resgate, incluindo custos de recuperação, tempo de inatividade e danos à reputação. \\
\hline
Percepção de ser a única ou a mais rápida opção quando os backups falharam ou foram comprometidos. & Pode marcar a organização como um alvo disposto a pagar, levando a futuros ataques. & 21\% das organizações que pagaram em 2023 ainda não conseguiram recuperar seus dados. \\
\hline
Evitar danos à reputação associados à perda de dados ou interrupção prolongada. & Implicações éticas e potenciais sanções legais por negociar com criminosos ou entidades sancionadas. & A notícia do pagamento do resgate pode, por si só, causar dano reputacional. \\
\hline
\end{tabular}
\end{table}
