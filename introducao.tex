% O que é
%  • Histórico
%  • Técnicas de ataque
%  • Os Ransomwares mais conhecidos – características
%  • Técnicas de proteção
%  • Como recuperar (quando é possível)

\section{Introdução}

O ransomware emergiu como uma das ameaças cibernéticas mais proeminentes e disruptivas da atualidade, afetando organizações de todos os tamanhos 
e setores, bem como usuários individuais. Caracterizado pela sua capacidade de negar acesso a sistemas ou dados críticos, exigindo um resgate para a sua 
restauração, o ransomware não só causa perdas financeiras diretas, mas também interrupções operacionais significativas, danos à reputação e potenciais 
violações de dados sensíveis. A sua evolução constante, desde os primeiros exemplares rudimentares até às sofisticadas variantes atuais que empregam 
táticas de extorsão múltipla e modelos de negócio como Ransomware-as-a-Service (RaaS), sublinha a necessidade de uma compreensão aprofundada desta ameaça. 
Este artigo tem como objetivo fornecer uma análise abrangente do ransomware, abordando a sua definição, histórico evolutivo, as técnicas de ataque empregadas, 
as famílias mais notórias, as estratégias de proteção e as metodologias de recuperação, complementada por dados e tendências recentes.





