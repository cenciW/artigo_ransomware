
\begin{table}[H] % Alterado para [H] para posicionamento "aqui"
    \centering
    \small
    \caption{Evolução Histórica do Ransomware}
    \label{tab:evolucao_historica}
    \sloppy % Permite quebras de linha mais flexíveis para evitar underfull hbox
    \begin{tabularx}{\textwidth}{|l|p{3cm}|p{5cm}|p{5cm}|}
        \hline
        \textbf{Período} & \textbf{Marco Principal} & \textbf{Características Notáveis} & \textbf{Impacto Significativo} \\ \hline
        1989 & AIDS Trojan (PC Cyborg) & Distribuição via disquetes, ocultação de diretórios, nomes de arquivos criptografados 
        (simples), resgate por correio postal. & Primeiro ransomware documentado; catalisador para leis de crimes cibernéticos. 
        \cite{CyberMaxxRansomwareHistory, Muniandy2024Ransomware, WatchGuardAIDSTrojan} \\ \hline
        Meados 2000s & Surgimento de \textit{Lockers} e \textit{Scareware} & Bloqueio de tela/sistema; falsos alertas de vírus; 
        criptografia simétrica fraca (e.g., GPCode). & Aumento gradual da atividade; experimentação com modelos de extorsão. 
        \cite{Tanni2022RedAlert, MasterDCRansomwareHowItWorks, Robb2024RansomwareHistory} \\ \hline
        2013 & CryptoLocker & Criptografia assimétrica forte (RSA); uso de Bitcoin para pagamento; disseminação via 
        \textit{phishing}. & Revolucionou o ransomware, tornando a recuperação quase impossível sem a chave; alta lucratividade. 
        \cite{CyberMaxxRansomwareHistory, Muniandy2024Ransomware, Robb2024RansomwareHistory} \\ \hline
        Fim dos 2010s & Ascensão do \textit{Ransomware-as-a-Service} (RaaS) & Plataformas oferecem ransomware a afiliados; 
        divisão de lucros; proliferação de ataques. & Redução da barreira de entrada para cibercriminosos; aumento massivo no 
        volume de ataques. \cite{CyberMaxxRansomwareHistory, Robb2024RansomwareHistory} \\ \hline
        2017 & WannaCry & Exploração da vulnerabilidade EternalBlue (SMB); propagação tipo \textit{worm}; impacto global em 
        larga escala. & Demonstrou a capacidade de interrupção massiva e a vulnerabilidade de sistemas não corrigidos. 
        \cite{CyberMaxxRansomwareHistory, WikipediaWannaCry} \\ \hline
        2020 - Presente & Predominância da Dupla e Tripla Extorsão & Exfiltração de dados antes da criptografia; ameaça 
        de vazamento público; ataques DDoS adicionais. & Aumenta a pressão sobre as vítimas; \textit{backups} sozinhos não 
        são suficientes. \cite{CyberMaxxRansomwareHistory, Robb2024RansomwareHistory, ThreatDownALPHVBlackCat, AkamaiBlackCatRansomware} \\ \hline
        2021 - Presente & Foco em Alvos de Alto Valor e Infraestruturas Críticas & Ataques direcionados a grandes 
        corporações, hospitais, governos; demandas de resgate milionárias. & Interrupções severas em serviços essenciais; 
        aumento do impacto econômico e social. \cite{CyberMaxxRansomwareHistory, Robb2024RansomwareHistory} \\ \hline
        2023 - Presente & Uso Crescente de IA e Exploração de Novas Vulnerabilidades & Desenvolvimento de ransomware assistido 
        por IA; exploração de vulnerabilidades em dispositivos IoT e \textit{firmware}. & Potencial para ataques mais sofisticados, 
        evasivos e automatizados. \cite{ENISA_ETL_2023, KasperskyRansomwareReport2025} \\ \hline
    \end{tabularx}
\end{table}